En base a los experimentos realizados, podemos concluír que el protocolo IPv4 se utiliza mucho más que el protocolo IPv6 incluso en redes locales.
También notamos que la proporción de paquetes de tipo ARP aumenta a medida que crece la red local. Sin embargo, en comparación a la totalidad de paquetes, siempre es baja,
ya que al crecer la red, también crecen los envíos de otros tipos de paquetes, incluso en mayor medida. Esto muestra que el protocolo ARP es eficiente y no sobrecarga la
red con paquetes de control.
En todos los casos, la dirección IP del router fue la que más paquetes manejó.
Creemos que la entropía es inversamente proporcional al control que se tiene sobre la red; aunque no tenemos suficiente evidencia para asegurarlo.
