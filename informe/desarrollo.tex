Para la implementación de las consignas pedidas se utilizó el lenguaje de programación \texttt{python}, tal como
fue recomendado por la cátedra. Para acceder a la placa de red se utilizó el paquete \texttt{scapy} que provee
funciones específicas para ello.

\subsection{Primera consigna: capturando tráfico}

\subsubsection{Ejercicio 1}
El código que implementa la herramienta que escucha pasivamente los paquetes Ethernet de la red es \texttt{e1.py}
y se encuentra en el directorio \texttt{src}. Toma como parámetro opcional un entero que se traduce en la cantidad
de segundos que va a permanecer activo. El valor por defecto es $10$ segundos. La función \texttt{main} del script
utiliza la función \texttt{sniff} del paquete \texttt{scapy} pasándole como parámetro de \texttt{timeout} el
parámetro ingresado (o $10$ si no se ingresó ninguno) y como parámetro de \texttt{prn} la función
\texttt{monitor\_callback} que toma un paquete de red y lo imprime mediante un llamado a la función \texttt{show}.
La forma de ejecutarlo es

\[
\texttt{\$ sudo python e1.py [TIMEOUT]}
\]

Notar que para ejecutarlo se necesitan permisos de administrador ya que la función \texttt{sniff} de \texttt{scapy}
necesita permisos para acceder a la placa de red.

\subsubsection{Ejercicio 2}
El código que implementa la herramienta para calcular la entropía de la fuente S en la red local es \texttt{e2.py}
y se encuentra en el directorio \texttt{src}. Este programa es una modificación del anterior, \texttt{e1.py}.
La principal modificación es que los paquetes obtenidos por el llamado a \texttt{sniff} ahora se almacenan para
operar sobre ellos luego. Se anula el parámetro \texttt{prn} pero se conserva el \texttt{timeout} (el cuál sigue
siendo un parámetro opcional del programa). Una vez recibidos todos los paquetes, se obtiene el tipo de cada uno
mediante el atributo \texttt{type}. En este punto encontramos que no todos los paquetes capturados poseen dicho
atributo, así que atrapamos una excepción al leer el tipo. En caso de saltar la excepción, consideramos que el
paquete tiene tipo $0x0000$ y lo imprimimos llamando a \texttt{monitor\_callback}. A través de la función de
mapeo \texttt{map\_number\_to\_name} convertimos a una cadena el valor del tipo del paquete. Esta función
utiliza un diccionario basado en la siguiente tabla\footnote{https://en.wikipedia.org/wiki/EtherType}:

\begin{tabular}{|r|l|c|r|l|c|r|l|}
\hline
0x0800 & IPv4 & & 0x8847 & MPLS Unicast & & 0x88CD & SERCOS III \\
\hline
0x0806 & ARP & & 0x8848 & MPLS Multicast & & 0x88E1 & HomePlug AV \\
\hline
0x0842 & WakeOn LAN & & 0x8863 & PPPoE Discovery & & 0x88E3 & MRP \\
\hline
0x22F3 & IETF TRILL & & 0x8864 & PPPoE Session & & 0x88E5 & MAC security \\
\hline
0x6003 & DECnet & & 0x8870 & Jumbo & & 0x88E7 & PBB \\
\hline
0x8035 & RARP & & 0x887B & HomePlug 1.0 & & 0x88F7 & PTP \\
\hline
0x809B & Ethertalk & & 0x888E & 802.1X & & 0x8902 & CFM \\
\hline
0x80F3 & AARP & & 0x8892 & PROFINET & & 0x8906 & FCoE \\
\hline
0x8100 & 802.1Q & & 0x889A & SCSI & & 0x8914 & FCoE Init \\
\hline
0x8137 & IPX & & 0x88A2 & ATA & & 0x8915 & RoCE \\
\hline
0x8204 & QNX Qnet & & 0x88A4 & EtherCAT & & 0x891D & TTE  \\
\hline
0x86DD & IPv6 & & 0x88A8 & 802.1ad & & 0x892F & HSR \\
\hline
0x8808 & EFC & & 0x88AB & Powerlink & & 0x9000 & ECTP \\
\hline
0x8819 & CobraNet & & 0x88CC & LLDP & & & \\
\hline
\end{tabular}

Luego se utiliza la función \texttt{Counter} de \texttt{python} para generar el diccionario \texttt{cantidades}
cuyas claves son los tipos de protocolos y sus respectivos valores son la cantidad de paquetes de tal tipo.
A partir de este diccionario (el cual se imprime para verificación) se calcula la probabilidad de cada tipo.
Finalmente se calcula la entropía de la fuente utilizando la probabilidad de cada tipo y se la imprime. La
forma de ejecutarlo es:

\[
\texttt{\$ sudo python e2.py [TIMEOUT]}
\]

\subsubsection{Ejercicio 3}
El código que implementa la herramientade distinción de nodos (hosts) de la red, basada únicamente en paquetes que
utilizan el protocolo ARP es \texttt{e3.py} y se encuentra en el directorio \texttt{src}. El criterio para la
diferenciación de los nodos que se encuentran en la red es ... %TODO

Una vez más, implementar este programa consistió en modificar el anterior, ya que la única diferencia entre ambas
consignas es el atributo de cada paquete utilizado como símbolo de la fuente. Al llamado a la función \texttt{sniff}
se le agregó el parámetro \texttt{filter=``arp''} para que sólo se examinen los paquetes \texttt{ARP}. En lugar de
obtener el tipo de cada paquete, lo que se obtiene es %TODO: ¿la ip fuente?¿la ip destino?
El resto del código es idéntico al de \texttt{e2.py}. La forma de ejecutarlo es:

\[
\texttt{\$ sudo python e3.py [TIMEOUT]}
\]

\subsection{Segunda consigna: gráficos y análisis}

Para la parte de experimentación se implementó otro script de python que realiza una medición y sobre
esa medición calcula la entropía para ambas fuentes. De esta forma se pueden comparar los dos análisis
realizados sobre una misma muestra, cosa que no habríamos podido si ejecutábamos primero un script y luego
el otro. El programa correspondiente es \texttt{sniffer.py} y se encuentra en el directorio \texttt{src}.

La mayor parte del código es idéntica a la de \texttt{e2.py} y \texttt{e3.py} combinados. Sin embargo,
también se aplicaron algunas optimizaciones. Por ejemplo, ya no se almacenan los paquetes capturados por
\texttt{sniff}, sino que se procesan a medida que se capturan. Para ello, se modificó la función
\texttt{monitor\_callback} de forma que obtenga el tipo (tal como se hizo en \texttt{e2.py}) y, en caso
de ser un paquete \texttt{ARP}, %TODO: ¿la ip fuente?¿la ip destino?
(tal como se hizo en \texttt{e3.py}) de cada paquete a medida que son capturados. Es por esto que se volvió
a la versión original de \texttt{sniff} pasándole como parámetro \texttt{prn=monitor\_callback}. Tras finalizar
la escucha, se generan los diccionarios, se imprimen y se calculan las entropías. Además, se escribe a un archivo
los valores de cada diccionario para poder ser graficados como histogramas con \texttt{gnuplot}. La forma de
ejecutarlo es:

\[
\texttt{\$ sudo python sniffer.py FILE\_PREFIX [TIMEOUT]}
\]

El parámetro de \texttt{timeout} sigue siendo opcional (10 segundos por defecto). Los archivos de salida
se componen del prefijo \texttt{FILE\_PREFIX} pasado como parámetro obligatorio y las cadenas \texttt{Protocolos},
\texttt{IpsSrcArp} o \texttt{IpsDstArp}, según corresponda. Estos archivos se guardan en la carpeta \texttt{mediciones}.

\subsubsection{Experimento 1: }

Este experimento consiste en ... %TODO

\subsubsection{Experimento 2: }

Este experimento consiste en ... %TODO

\subsubsection{Experimento 3: }

Este experimento consiste en ... %TODO
